\chapter{Manual de usuario}

\section{Instalación}
Para utilizar AutoSmart, sigue estos pasos:
\begin{itemize}
    \item Descarga el archivo APK desde el repositorio de GitHub del proyecto.
    \item Instala la aplicación en un dispositivo Android 7.0 o superior.
    \item Concede los permisos necesarios para notificaciones y almacenamiento cuando la aplicación lo solicite.
\end{itemize}

\section{Primeros pasos}
Al abrir la aplicación por primera vez, el usuario debe registrarse con un correo electrónico y una contraseña segura. Tras el registro, podrá acceder a la pantalla principal de la aplicación.

\section{Gestión de vehículos}
Desde el menú principal, el usuario puede añadir un nuevo vehículo pulsando el botón correspondiente. Se solicitarán datos como marca, modelo, año y matrícula. Los vehículos pueden ser editados o eliminados en cualquier momento.

\section{Gestión de mantenimientos}
Para cada vehículo, el usuario puede registrar mantenimientos indicando la fecha, el tipo de servicio, una breve descripción y el coste. Es posible editar o eliminar mantenimientos existentes desde el historial.

\section{Notificaciones}
En la sección de configuración, el usuario puede activar o desactivar las notificaciones de mantenimiento. Cuando un mantenimiento programado se acerca, la aplicación enviará un recordatorio automático.

\section{Historial y estadísticas}
El usuario puede consultar el historial de mantenimientos de cada vehículo, así como visualizar los gastos acumulados y filtrar por tipo de servicio. Esta funcionalidad ayuda a llevar un control económico y a planificar futuros mantenimientos.

\section{Recomendaciones}
\begin{itemize}
    \item Mantén actualizados los datos de tus vehículos y mantenimientos para recibir notificaciones precisas.
    \item Realiza copias de seguridad periódicas si la aplicación lo permite.
    \item Consulta el manual de usuario integrado en la app para resolver dudas frecuentes.
\end{itemize}

AutoSmart está diseñada para ser intuitiva y accesible, facilitando la gestión del mantenimiento de vehículos y contribuyendo a la seguridad y el ahorro del usuario. 