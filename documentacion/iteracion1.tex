\chapter{Primera iteración}

En la primera iteración del desarrollo de AutoSmart se implementó la funcionalidad básica de la aplicación, sentando las bases para el resto del proyecto. El objetivo principal de esta fase fue obtener una versión funcional mínima (MVP) que permitiera validar la idea y recoger feedback de usuarios reales.

Durante esta iteración se desarrollaron los siguientes módulos:
\begin{itemize}
    \item \textbf{Gestión de usuarios:} Registro, autenticación y edición de perfil, garantizando la privacidad y seguridad de los datos.
    \item \textbf{Gestión de vehículos:} Alta, edición y baja de vehículos, permitiendo almacenar información relevante como marca, modelo, año y matrícula.
    \item \textbf{Gestión de mantenimientos:} Registro, edición y eliminación de mantenimientos asociados a cada vehículo, con campos para fecha, tipo, descripción y coste.
    \item \textbf{Notificaciones:} Configuración y envío de recordatorios automáticos para los próximos mantenimientos, ayudando al usuario a no olvidar tareas importantes.
\end{itemize}

Se realizaron pruebas funcionales y de usabilidad con un grupo reducido de usuarios, lo que permitió detectar mejoras en la interfaz y en la lógica de notificaciones. El feedback recibido fue fundamental para priorizar nuevas funcionalidades y optimizar la experiencia de usuario en las siguientes iteraciones. 