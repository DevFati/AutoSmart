\chapter{Arranque del proyecto}

El arranque del proyecto AutoSmart comenzó con la identificación de una necesidad real en el entorno personal y familiar de la autora: la dificultad para recordar y gestionar los mantenimientos periódicos de los vehículos. Esta motivación personal se complementó con una investigación sobre el sector de la automoción y las soluciones tecnológicas existentes en el mercado.

Se definieron los objetivos generales y específicos del proyecto, priorizando la usabilidad, la gratuidad y la adaptación a la normativa española. Se analizaron aplicaciones similares, identificando sus limitaciones y oportunidades de mejora.

En esta fase se seleccionaron las tecnologías a emplear: Java como lenguaje principal por su robustez y compatibilidad con Android, Firebase para la gestión de datos en la nube y Room (SQLite) para el almacenamiento local. También se optó por seguir las guías de Material Design para garantizar una experiencia de usuario moderna y accesible.

El arranque del proyecto sentó las bases para un desarrollo estructurado, orientado a la resolución de problemas reales y a la adquisición de competencias profesionales en el ámbito del desarrollo de aplicaciones multiplataforma. 