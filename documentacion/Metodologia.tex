\chapter{Metodología}

Para el desarrollo de AutoSmart se ha seguido una metodología ágil, basada en iteraciones cortas y entregas incrementales, lo que ha permitido adaptarse a los cambios y mejorar el producto de forma continua. Esta elección metodológica se justifica por la necesidad de responder rápidamente a los problemas detectados y a las sugerencias de los usuarios durante el desarrollo.

Se han empleado herramientas de control de versiones (GitHub) para gestionar el código fuente y Trello para la organización de tareas y seguimiento del avance del proyecto.

El proceso de desarrollo se ha estructurado en las siguientes fases:
\begin{enumerate}
    \item Análisis de requisitos y estudio de mercado: identificación de necesidades, análisis de la competencia y definición de funcionalidades clave.
    \item Diseño de la arquitectura y la base de datos: elaboración de diagramas de casos de uso, entidad-relación y clases, y definición de la estructura del software.
    \item Implementación de la funcionalidad principal: desarrollo de los módulos de gestión de usuarios, vehículos, mantenimientos y notificaciones.
    \item Pruebas y validación: realización de pruebas funcionales, de usabilidad y de rendimiento, así como validación con usuarios reales.
    \item Documentación y presentación: redacción de la documentación técnica y del manual de usuario, y preparación de la defensa del proyecto.
\end{enumerate}

Esta metodología ha permitido mantener una visión global del proyecto, priorizar tareas y garantizar la calidad del producto final. 