\chapter{Conclusiones}

El desarrollo de AutoSmart ha supuesto un reto técnico y personal que ha permitido a la autora adquirir y consolidar competencias en el ámbito del desarrollo de aplicaciones multiplataforma, la gestión de bases de datos y la experiencia de usuario. El proyecto ha logrado cubrir una necesidad real en la gestión de mantenimientos de vehículos, aportando una solución gratuita, intuitiva y adaptada al mercado español.

Entre las principales aportaciones del proyecto destacan:
\begin{itemize}
    \item La creación de una herramienta accesible y útil para particulares y empresas, que contribuye a la seguridad vial y al ahorro económico.
    \item La integración de notificaciones automáticas y un historial digital, facilitando el control y la planificación de los mantenimientos.
    \item El diseño de una interfaz moderna y adaptada a las necesidades de los usuarios españoles.
\end{itemize}

\section*{Desafíos y resolución de problemas}
Durante el desarrollo de AutoSmart, se presentaron diversos desafíos técnicos y de diseño, entre los que destacan:
\begin{itemize}
    \item \textbf{Gestión de notificaciones en Android:} Fue necesario investigar y adaptar el sistema de notificaciones para garantizar su funcionamiento en diferentes versiones del sistema operativo.
    \item \textbf{Validación de fechas y horas:} Se implementaron controles y mensajes de error claros para evitar registros inválidos y mejorar la experiencia de usuario.
    \item \textbf{Persistencia y sincronización de datos:} La integración de Room (SQLite) y Firebase supuso un reto para mantener la coherencia y seguridad de la información.
    \item \textbf{Diseño adaptativo:} Se realizaron pruebas en distintos dispositivos para asegurar la correcta visualización y usabilidad de la aplicación.
\end{itemize}

Superar estos desafíos ha permitido a la autora desarrollar habilidades de resolución de problemas, investigación y adaptación a nuevas tecnologías.

\section*{Mejoras futuras}
Como líneas de mejora y evolución del proyecto, se plantean:
\begin{itemize}
    \item Añadir soporte multiplataforma (iOS, web) para llegar a un mayor número de usuarios.
    \item Integrar recordatorios de ITV y seguro, así como la posibilidad de exportar el historial a PDF.
    \item Incorporar gráficos de gastos y estadísticas avanzadas.
    \item Mejorar la accesibilidad y la personalización de la interfaz.
\end{itemize}

En definitiva, AutoSmart representa una contribución significativa al ámbito de la movilidad y la gestión digital, y sienta las bases para futuros desarrollos y aprendizajes en el campo de la ingeniería del software. 