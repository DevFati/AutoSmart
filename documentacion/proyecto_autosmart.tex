\documentclass[12pt,a4paper]{book}

% Paquetes básicos
\usepackage[spanish]{babel}
\usepackage[utf8]{inputenc}
\usepackage[T1]{fontenc}
\usepackage{graphicx}
\usepackage{mathpazo}
\usepackage{xcolor}
\usepackage{titlesec}
\usepackage{tcolorbox}
\usepackage{booktabs}

% Definición de colores institucionales
\definecolor{usred}{RGB}{153,0,51}
\definecolor{uscream}{RGB}{255,243,207}
\definecolor{usgray}{RGB}{245,245,245}

% Estilo decorativo para capítulos
\titleformat{\chapter}[display]
  {\normalfont\bfseries\color{usred}\centering}
  {\color{usred}\rule{\textwidth}{3pt}\\[1.5cm]
   \fontsize{30}{36}\selectfont\thechapter}
  {0pt}
  {\Huge\bfseries\color{usred}}
  [\vspace{1.5cm}\color{usred}\rule{\textwidth}{3pt}]

% Estilo decorativo para secciones
\titleformat{\section}
  {\normalfont\Large\bfseries\color{usred}}
  {\thesection}{1em}{}
  [\color{usred}\rule{0.2\textwidth}{1.5pt}]

% Estilo decorativo para subsecciones
\titleformat{\subsection}
  {\normalfont\large\bfseries\color{usred}}
  {\thesubsection}{1em}{}
  [\color{usred}\rule{0.1\textwidth}{1pt}]

% Estilo decorativo para tablas
\usepackage{array}
\newcolumntype{L}[1]{>{\raggedright\arraybackslash}p{#1}}
\newcolumntype{C}[1]{>{\centering\arraybackslash}p{#1}}
\newcolumntype{R}[1]{>{\raggedleft\arraybackslash}p{#1}}

% Cuadros de texto decorativos
\tcbset{
  highlightbox/.style={colback=uscream!80!white, colframe=usred, fonttitle=\bfseries, coltitle=usred, boxrule=1pt, arc=2mm, left=2mm, right=2mm, top=1mm, bottom=1mm}
}

\begin{document}

% --- PORTADA DECORATIVA ---
\begin{titlepage}
    \thispagestyle{plain}
    \pagenumbering{roman}
    \setcounter{page}{1}
    \centering
    % Caja de título
    \vspace*{2cm}
    \noindent
    \colorbox{uscream}{
        \parbox{0.85\textwidth}{
            \centering
            \vspace{0.5cm}
            {\bfseries\LARGE\color{usred} AutoSmart: Aplicación de Gestión de Mantenimientos de Vehículos\\[0.2cm]}
            \vspace{0.5cm}
        }
    }
    \vspace{1.5cm}

    % Nombre del alumno y datos
    {\Large\color{usred}\textbf{Fatima Mortahil Chachou}}\\[0.5cm]
    {\large\color{usred}Trabajo Fin de Grado}\\[0.2cm]
    {\large\color{usred}Supervisado por Dr. David Arriero Ollero}\\[2cm]

    % Logo
    \includegraphics[width=0.25\textwidth]{logo_us.png}\\[1.5cm]

    % Universidad y fecha
    {\large\color{usred}Universidad de Sevilla}\\[0.5cm]
    {\large\color{usred}\today}
    \vfill
    % Número de página romano
    {\color{usred}\textbf{\thepage}}
\end{titlepage}

% --- RESUMEN DECORATIVO ---
\newpage
\thispagestyle{plain}
\setcounter{page}{2}
\noindent
\color{usred}\rule{\textwidth}{3pt}
\vspace{1.5cm}

\begin{center}
    {\LARGE\bfseries\color{usred} RESUMEN}
\end{center}

\vspace{1.5cm}

\begin{center}
    \begin{minipage}{0.8\textwidth}
        \centering
        Aquí va tu resumen: un párrafo sobre el problema planteado en el proyecto y la solución. Máximo 300 palabras.
    \end{minipage}
\end{center}

\vfill
\color{usred}\rule{\textwidth}{3pt}

\begin{center}
    {\color{usred}\textbf{\thepage}}
\end{center}

% --- ÍNDICE GENERAL DECORATIVO ---
\newpage
\thispagestyle{plain}
\setcounter{page}{3}
\noindent
\color{usred}\rule{\textwidth}{3pt}
\vspace{1.5cm}

\begin{center}
    {\LARGE\bfseries\color{usred} ÍNDICE GENERAL}
\end{center}

\vspace{1cm}

\tableofcontents

\vfill
\color{usred}\rule{\textwidth}{3pt}

\begin{center}
    {\color{usred}\textbf{\thepage}}
\end{center}

% --- CAMBIO A NUMERACIÓN ARÁBIGA ---
\newpage
\pagenumbering{arabic}
\setcounter{page}{1}

% Ejemplo de capítulo decorativo
\chapter{Introducción}
Aquí comienza el contenido de tu TFG con el estilo decorativo de capítulos.

\section{Ejemplo de sección decorativa}
Texto de la sección.

\subsection{Ejemplo de subsección decorativa}
Texto de la subsección.

% Ejemplo de tabla decorativa
\begin{table}[h!]
\centering
\rowcolors{1}{usgray}{white}
\begin{tabular}{L{4cm}C{3cm}R{3cm}}
\toprule
\rowcolor{uscream}
\textbf{Columna 1} & \textbf{Columna 2} & \textbf{Columna 3} \\
\midrule
Dato 1 & Dato 2 & Dato 3 \\
Dato 4 & Dato 5 & Dato 6 \\
\bottomrule
\end{tabular}
\caption{Ejemplo de tabla decorativa}
\end{table}

% Ejemplo de cuadro decorativo
\begin{tcolorbox}[highlightbox, title={Nota importante}]
Este es un cuadro decorativo para resaltar información relevante en el TFG.
\end{tcolorbox}

% Puedes seguir con tus capítulos y apartados...

\section{Estructura funcional del sistema}
AutoSmart está diseñado para ser intuitivo y eficiente, permitiendo al usuario gestionar todos los aspectos relacionados con el mantenimiento de sus vehículos. El sistema se divide en varios módulos:

\begin{itemize}
    \item \textbf{Gestión de usuarios:} Permite el registro, autenticación y edición del perfil del usuario. Cada usuario tiene acceso privado a sus datos y vehículos.
    \item \textbf{Gestión de vehículos:} El usuario puede añadir, editar o eliminar vehículos. Cada vehículo almacena información relevante como marca, modelo, año y matrícula.
    \item \textbf{Gestión de mantenimientos:} Para cada vehículo, el usuario puede registrar mantenimientos, especificando fecha, tipo, descripción y coste. También puede editar o eliminar registros existentes.
    \item \textbf{Historial y estadísticas:} El usuario puede consultar el historial de mantenimientos de cada vehículo, visualizar los gastos acumulados y filtrar por tipo de servicio.
    \item \textbf{Notificaciones:} El sistema permite configurar recordatorios automáticos para los próximos mantenimientos, avisando al usuario mediante notificaciones push.
    \item \textbf{Configuración y personalización:} El usuario puede ajustar sus preferencias, como la frecuencia de notificaciones, el tema visual de la app y la privacidad de sus datos.
\end{itemize}

Esta estructura modular facilita el mantenimiento y la escalabilidad de la aplicación, permitiendo añadir nuevas funcionalidades en el futuro. 
\end{document} 